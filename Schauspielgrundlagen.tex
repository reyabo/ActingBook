\documentclass[ngerman, a4paper, twoside]{scrbook}%scrartcl; report; scrbook
\usepackage[german,ngerman]{babel}
\usepackage[utf8]{inputenc}
\usepackage[T1]{fontenc}

\author{Oliver Bayer}
\title{Schauspielgrundlagen - Praktische Übungen für Schauspieler/innen}
\usepackage{amsmath}
\usepackage{amsfonts}
\usepackage{amssymb}
\usepackage{hyperref}


\begin{document}
\frontmatter
\maketitle
%\newpage
\tableofcontents
%\newpage
	\chapter{Vorwort}
	Dieses Buch stellt eine Sammlung von unterschiedlichen Übungen zur Erarbeitung einer Rolle oder zum Anleiten einer Schauspielgruppe in Schulen oder Vereinen dar. Neben unterschiedlichen grundlegenden Übungen und Spielen, werde ich auch die drei bekanntesten Schauspielmethoden vorstellen und zwar mit den Übungen, die ich während meiner Ausbildung kennengelernt und erfahren haben.\\
	Ich weiß auch noch nicht genau welche Einteilung innerhalb des Kapitel Schauspiel-Methoden sinnvoll ist. Momentan ist das alles noch etwas chaotisch, ich werde aber versuchen das ganze zu vereinheitlichen.

	\mainmatter
	\part{Grundlagen}
	\chapter{Grundlagen}
	Unabhängig davon weshalb man Theater spielen möchte und auch unabhängig von der Größe der Bühne auf der Stück letzten Endes aufgeführt werden soll, gibt es einige Grundlagen (zumindest in meinen Augen), die sowohl Laien als Profis beim Spielen beachten sollten. Mit diesen Grundlagen ist zwar noch nicht gesagt, dass ein \emph{authentisches Verkörpern}
	 erreicht wird, aber zumindest könnten sie dazu verhelfen, dass das Spiel organischer und echter wirkt, folglich zum \emph{authentischen Spiel} %\footnote{Im Verlauf des Buches werden die Begriffe authentisches Spielen bzw. Verkörpern öfters vorkommen. Ob es in bei den Profis dabei eine Unterscheidung gibt, weiß ich noch nicht. Für mich allerdings schon. Authentisches Spielen ist in meinen Augen immer möglich. }
	 wird.
	\section{Not}
	Einer der wichtigsten Begriff innerhalb des Schauspielens ist die \emph{Not} \footnote{Manchmal auch als Etüde, Willen o.ä bezeichnet. Im Verlauf des Buches werde ich nur noch den Begriff der \emph{Not} benutzen.}. Die \emph{Not} des Schauspielers wirkt sich unmittelbar auf das Spielen und auf das Sprechen des Textes aus. Grundsätzlich ist die \emph{Not} der Antrieb der Figur. Ihr Grund weshalb sie spricht und handelt. Die \emph{Not} sollte dabei möglichst nah am Text/ Stück  sein. Ohne die \emph{Not} ist kein authentisches Spielen möglich.

	\section{Persönliche Not}
	Die \emph{persönliche Not}
	\footnote{Beim Anwenden der persönlichen Not nicht zu tief graben und nur Konflikte benutzten, die man selbst abgeschlossen und verarbeitet hat.}
	ist eine Grundvoraussetzung für das \emph{authentische Verkörpern} und liefert die emotionale Komponente für die Rolle.  Sie hat nichts mehr mit der Rolle oder dem Text zu tun, sondern nur noch mit dem Schauspieler selbst. Häufig erwächst sie aus inneren oder zwischenmenschlichen Konflikte, die  den Darstellenden zu dem Menschen gemacht haben, der er heute ist. \\
	Die \emph{persönliche Not} hilft dem Darstellenden die \emph{Not} zu seiner eigenen zu machen. Sie sollte am besten aufgeschrieben werden. Es kommt dabei nicht darauf an, den Text des Stückes umzuschreiben, sondern seine eigene persönliche Sicht/ Interpretation
	\footnote{ohne, dass sie begründet oder sinnvoll sein muss}
	 des Textes festzulegen.
	 \vfill
	 \section{Textunabhängige Handlung}
	 Nachdem man sowohl die \emph{persönliche} als auch die "`normale"' \emph{Not} gefunden hat, gilt es nun den Text so zu sprechen, dass er natürlich und nicht aufgesagt wirkt. 	 Dazu gibt es die Übung der \emph{textunabhängigen Handlung}.\\
	 Der Darstellende sucht sich eine Handlung, die er beiläufig aber mit einer gewissen Not erledigen kann und versucht nun darauf den Text zu sprechen.\\
	 \emph{Textunabhängige Handlungen} können sein:
	 \begin{itemize}
	 	\item Brille putzen
	 	\item Kartoffeln o.ä. schälen
	 	\item Geldscheine zählen
	 	\item Staubsaugen
	 	\item Abwaschen
	 	\item Aufräumen
	 	\item Haare zupfen
	 	\item
	 \end{itemize}



	\setcounter{chapter}{0}
	\setcounter{section}{0}
	\part{Schauspiel-Methoden}
	\chapter{Tschechow}
	Tschechow geht davon, dass bestimmte Haltungen (Gesten) gewisse Gefühle im Menschen auslösen, dabei will er den Rahmen des Wahrnehmbaren stetig erweitern. Er arbeitet mit seinem Ansatz also von "`außen nach innen"'. Über die Körperlichkeit soll ein Gefühl ausgelöst werden und dadurch ein authentisches Spiel entstehen. Diese Haltungen nennt er \emph{psychologische Geste}. Durch das Verknüpfung einer spezifischen Emotion mit einer \emph{psychologischen Geste} kann die psychische Gesundheit des Schauspielers gewahrt werden. Durch bestimmte Übungen gewinnt der Schauspieler einen Zugang zu seinem \emph{energetischen (psychologischen)} Körper. Diesen kann er zusammen oder unabhängig von seinem \emph{physischen Körper} bewegen. Das Ausdehnen und Zusammenziehen des \emph{energetischen Körpers} bis zu einer bestimmten Grenze nennt Tschechow den \emph{künstlerischen Rahmen.} Tschechow unterscheidet dabei zwischen \emph{Handlungen} und \emph{Aktivitäten}. Schaut man nur Aktivitäten zu, dann wird es auf der Bühne schnell langweilig.
	\section{Grundlage}

	\subsection{Sinne erweitern}
	Zunächst wird ein bestimmter Sinn erweitert. Dabei soll der Sinn zunächst noch im Körper erweitert werden und schließlich über ihn hinaus. Am Ende kann ein Gegenstand mit Hilfe dieses erweiterten Sinnes betrachtet werden. \\
	Danach soll der Sinn wieder zurück genommen werden und der gleiche Gegenstand erneut betrachtet werden. Dabei sind die Unterschiede festzustellen.\\
	\textbf{Beispiel: Geschmackssinn}:\\
	Den Geschmackssinn zunächst im Mund erweitern. Seinen eigenen Mund schmecken, die Zähne, den Mundraum, usw. Schließlich die Lippen und am Ende die Luft schmecken. Dann den Gegenstand holen und betrachten.
	\subsection{Denken spielen}
	Ähnlich wie die Sinne und der Körper erweitert werden kann, kann auch das Denken durch Erweiterung gespielt werden.\\
	Zunächst legt man seine Hand auf die Schädeldecke und versucht den Innenraum zu fühlen. Anschließend konzentriert man sich auf seine Augen und erweitert sein linkes/ rechtes Auge bzw. zieht es zusammen.
	\subsection{Fensterscheibe}
	Um Trauer oder Tränen auf der Bühne verkörpern zu können, geht Tschechow nicht den Weg des emotionalen Gedächtnisses. Er bedient sich der Imagination des Darstellers. Dazu stellt sich dieser eine Fensterscheibe vor, an der Regentropfen entlang laufen. Der Darsteller beobachtet die Regentropfen und atmet dabei tief ein und aus.
	\subsection{Emotionen deckeln}
	Mit Hilfe der \nameref{subsec:Wippe} können Emotionen gedeckelt werden. D. h. emotionale Ausbrüche können langsam aufgebaut werden oder spontan auf 100\% hochgefahren werden ohne einen riesigen Vorlauf zu haben.


	\vspace{1cm}

	\section{Bewegungsarbeit}
	\subsection{Ausladende Bewegungen}
	Teilnehmer führen große, raumfüllende und ausladende Bewegungen aus. Schließlich kann mit Hilfe folgender Vorstellungen die Bewegungsqualität verändert werden.
	\begin{itemize}
		\item Die Teilnehmer bewegen sich, als müssten sie ihren Körper durch Ton bewegen und verharren am Ende jeder Bewegung kurz darin.
		\item Die Teilnehmer bewegen sich wie durch Wasser. Gleichzeitig sind auch ihre Bewegungen fließend.
		\item Die Teilnehmer bewegen sich wie durch Luft. Gleichzeitig sind auch ihre Bewegungen fliegend und leicht.
	\end{itemize}
	\subsection{Staccato (Anfang und Ende)/ Legato (Fließend)}
	Diese Übung stellt eine Abfolge von sechs Bewegungsrichtungen (rechts links, oben, unten, vorne, hinten) dar, die nach einander wiederholt werden. Dabei wird jede Richtung bis zur größtmöglichen Spannung/ Grenze ausgeführt. Am Ende dieses maximalen Punktes muss der Stand noch sicher und fest sein. Dann werden zwei imaginierte Tennisbälle in die jeweilige Richtung geworfen. Dies erhöht den maximalen Spannungspunkt noch einmal. Bei der Übung muss darauf geachtet werden, dann alle Körperteile erst nach und nach in die Richtung bewegt werden und die Spannung erzeugen.
	\subsection{Blume}
	Teilnehmer stellen sich vor ein Samenkorn zu sein. Dazu machen sie sich ganz klein. Nun erblüht das Samenkorn langsam zu einer Blume und verwelkt schließlich wieder zu Erde. Nach dem Züricher Ressourcenmodell lässt sich die Übung und die Energie, die dabei entsteht von einer Makrogeste (Blume) auf eine Mikrogeste (Faust öffnen und schließen) übertragen, dadurch soll der gleiche Effekt erzielt werden nur mit einer kleineren Geste.
	\subsection{Wippe}  \label{subsec:Wippe}
	Hüftbreit hinstellen und anfangen vor und zurück zu pendeln. Dabei zunächst nur soweit, dass die Füße den Kontakt zum Boden nicht verlieren. Trotzdem versuchen bis an die Grenzen zu kommen. Sobald man sich etwas "`eingependelt"' hat, erweitert man die Bewegung mit einem Ausfallschritt nach vorne/ hinten. Auch hier den Ausfallschritt immer größer werden lassen und solange wiederholen, bis man sich sicher fühlt. Abschließend lässt man sich komplett fallen. Dabei fängt man den Fall nach vorne mit den Armen ab und nach hinten rollt man sich über den Po ab.
	\subsection{Energiezentrum "`Drosselgrube"'}
	Nach Tschechow liegt eines der Energiezentren zwischen den Schlüsselbeinen in der Drosselgrube. Jegliche Bewegung wird von hier aus gesteuert.
	\subsection{Winken aus der Drosselgrube}
	Zunächst wird die Drosselgrube lokalisiert und mit dem Finger warm gerieben. Anschließenden stellen sich die Teilnehmer vor, dass sie einer Person am Ende des Raums zu winken. Die Person sieht das aber nicht und die Teilnehmer winken energischer. Da die Person immer noch nicht reagiert winken alle ein Drittes mal. Dabei soll darauf geachtet werden, dass die Bewegung aus der Drosselgrube heraus erfolgt.\\
	Zum Vergleich kann anschließend die Übung wiederholt werden und das Energiezentrum verschoben werden (Bauch, Leisten).
	\section{Zusammenziehen und Ausdehnen} \label{sec:Zusammenziehen und Ausdehnen}
	Teilnehmer machen sich klein (zusammenziehen), am besten Embryostellung, aber wichtig ist, dass man es bequem hat. Man versucht so wenig Platz wie möglich einzunehmen. Nun beginnen die Teilnehmer sich auszudehnen. Sie versuchen so viel Platz wie nur möglich einzunehmen. Hat der Teilnehmer seine "`maximale Ausdehnung"' erreicht, zieht er sich wieder zurück. Wichtig ist in beiden Phasen bewusste Endpunkte zu setzen. Diese Übung zunächst im Liegen beginnen und schließlich im Stehen beenden. Hat man das Gefühl für das Zusammenziehen und Ausdehnen verinnerlicht, reicht es nur noch den inneren Körper zusammenzuziehen und auszudehnen und übertragen dabei das Gefühl auf das Ballen der Faust.
	\subsection{Hände schütteln mit zusammenziehen oder ausdehnen}
	Teilnehmer gehen durch den Raum und ziehen sich zunächst mit Hilfe der Faust zusammen und dehnen sich wieder aus. Irgendwann können sie anfangen sich gegenseitig mit Händeschütteln zu begrüßen und dabei den inneren Körper zusammenziehen bzw. ausdehnen.\footnote{In verschiedenen Durchgängen.}
	\section{Emotionales Gedächtnis}
	\subsection{Arm heben - physisch - psychisch}
	Teilnehmer strecken ihren rechten/ linken Arm soweit wie möglich nach oben. Dabei ins Extreme gehen. Allerdings bleiben beide Beine fest auf dem Boden.\\
	Nachdem jeder Teilnehmer den Arm einige Male nach oben gestreckt hat, denkt er nur noch die Bewegung und hebt dabei seinen Arm nicht mehr.\\
	Nachdem jeder Teilnehmer die Bewegung einige Male gedacht hat, wird nun beides verbunden. Zunächst wird die Bewegung gedacht und schließlich ausgeführt. Dabei müssen gedachter (\emph{psychischer} und echter \emph{physischer} Arm nicht in Einklang sein, sondern können bewusst von einander separiert werden. \\
	\textbf{Anmerkung:} Auch das Absenken des Arm sollte diesen Prozess durchlaufen.
	\subsubsection{Punkt im Raum berühren - Appell}
	Jeder Teilnehmer fixiert einen Punkt am anderen Ende des Raums und streckt die Hand danach aus. Soweit es geh, bleibt aber mit beiden Beinen auf dem Boden. Nun stellen man sich vor, dass aus dem physischen Arm ein psychischer Arm herausstrahlt, der den Punkt berühren kann. Dieser Arm, kann auch die komplette anliegende Fläche des Raums abtasten und streicheln.\\
	Nach gewisser Zeit wird die Übung nur in der Vorstellung wiederholt.
	\subsection{Zeitstrahl}
	Die Teilnehmer stellen sich vor, dass sie ihre Hand nach dem Horizont ausstrecken (Meer, Gebirge). Beide Beine bleiben auf dem Boden, trotzdem versuchen ins Extrem zu gehen.\\
	Irgendwann wächst ein psychischer Arm aus dem physischen heraus und greift nach dem Horizont, er greift weiter in den nächsten Tag, Woche und schließlich ins nächste Jahr.\\ Nun wird der andere Arm nach hinten ausgestreckt ohne ihm nachzublicken oder die Spannug für den vorderen zu verlieren. Auch aus dem hinteren Arm strahlt ein psychischer Arm heraus in die Vergangenheit. Zunächst einen Tag, dann eine Woche und schließlich einen Monat. \\
	Die Teilnehmer senken die Arme und rufen noch einmal die Erinnerung an das Ausstrecken der Arme hervor und gehen nun zwei kleine Schritte in die Zunkuft/ Vergangenheit.
	\subsection{Augen in den Schultern}
	Die Teilnehmer stellen sich vor, dass sie Augen in den Schulterblättern hätten und erkunden den Raum. Vorsichtig mit geschlossenen Augen rückwärts laufen. Ab und zu innen halten und sich dreidimensional von außen im Raum wahrnehmen. \\
	Später können die Augen geöffnet werden. Der Raum wird nun mit den echten und den gedachten Augen wahrgenommen und ab und an auch von außen.
	\subsection{3-Wege}
	Teilnehmer denken sich drei Bewegungen aus. Nun sagen sie die erste Bewegung, denken sie sich und führen sie abschließend aus. Dann wird dies mit der zweiten und dritten wiederholt.\\

	\textbf{Variante:} \vspace{-.3cm}
	\begin{itemize}
		\item ohne Sagen $\rightarrow$ nur denken und ausführen
		\item mit Bewegungsqualität (Ton, Wasser, Luft)
		\item mit Subtext/ Appell
	\end{itemize}

	\section{Archetypen}\label{sec:Archetypen}
	Tschechow hat diese Archetypen (Stab, Ball, Schleier) an Shakespeare Figuren gebunden.
	\subsection{Bild des Stabs einverleiben} \label{subsec:Bild des Stabs einverleiben}
	Teilnehmer sammeln Besonderheiten eines Stabs (fest, unflexibel, usw.) und gehen als Stab durch den Raum. Dabei kommt es nicht darauf an pantomimisch einen Stab darzustellen, sondern den Stab zu verkörpern. Dabei kann zunächst im Liegen oder Sitzen angefangen werden, irgendwann sollte aber zum Gehen übergegangen werden. Nach und nach wird nun das physische Gefühl zu einem inneren Bild abgelegt.
	\subsection{Haus des Stabs} \label{subsec:Haus des Stabs}
	Wenn die Verkörperung des Stabes geklappt hat, ist der Teilnehmer im "`Haus des Stabes"' angekommen. Er kann nun einen spezifischen Stab verkörpern. Die Übung folgt dabei der oben beschriebenen. Der \emph{Stab} ist ein sehr kopflastiger Archetyp.\\
	Sobald ein spezifischer Stab verkörpert wurde, können die Stäbe präsentiert werden.\\
	Abschließend kann nun die Stab-Figur (Hamlet) mit der bekannten Monologzeile "`Sein oder nicht sein, das ist hier die Frage"' verkörpert werden.
	\subsection{Bild des Ball einverleiben}
	Wie \ref{subsec:Bild des Stabs einverleiben} \nameref{subsec:Bild des Stabs einverleiben}
	\subsection{Haus des Balls}
	Wie \ref{subsec:Haus des Stabs} \nameref{subsec:Haus des Stabs}. Der Ball ist ein impulsiver Archetyp. Er entscheidet aus dem Bauch heraus\\
	Hier kann nun Romeo verkörpert werden. "`Sie ist es. Meine Göttin, meine Liebe. Oh wüsste sie das sie es ist.
	\subsection{Bild des Schlleier einverleiben}
	Wie \ref{subsec:Bild des Stabs einverleiben} \nameref{subsec:Bild des Stabs einverleiben}. Tipp: Mit den Ecken des Schleiers anfangen und dann dem Körper vertrauen.
	\subsection{Haus des Schleiers}
	Wie \ref{subsec:Haus des Stabs}. \nameref{subsec:Haus des Stabs}.\\
	Der \emph{Schleier} ist ein emotionaler \emph{Archetyp}. Die Grenzen zwischen \emph{Ball} und \emph{Schleier} sind sehr vage in meinen Augen.\\
	Hier kann nun Julia verkörper werden.

	\vspace{0.5cm}

	\subsection{Raumlauf mit "`Ich will …"'} \label{subsec:Ich will}
	Teilnehmer laufen durch den Raum und sagen den Satz "`Ich will …"' Der Satz muss in der Vorstellung beendet werden. Es existiert also ein richtiges Wollen. Dabei sollte das Wollen sehr stark und mächtig sein.\\
	Diese Übung lässt sich auf mit dem Ball verbinden.\\
	\subsection{Raumlauf mit "`Ich lehne … ab"'}
	siehe \ref{subsec:Ich will} \nameref{subsec:Ich will}. \\
	Diese Übung lässt sich ebenfalls mit dem Ball verbinden.
	\subsection{Raumlauf mit "`Ich denke …"'}
	siehe \ref{subsec:Ich will} \nameref{subsec:Ich will}.\\
	Diese Übung lässt sich mit dem Stab verbinden.\\

	\vspace{1cm}
	\subsubsection{Weitere Archetypen}
	\begin{itemize}
		\item Raumlauf mit "`Ich bin ein Opfer"'\\
			siehe \ref{subsec:Ich will} \nameref{subsec:Ich will}.
		\item Raumlauf mit "`Ich bin ein König"'\\
		siehe \ref{subsec:Ich will} \nameref{subsec:Ich will}.
		\item Raumlauf mit "`Ich bin ein Sohn/ eine Tochter"'\\
			siehe \ref{subsec:Ich will} \nameref{subsec:Ich will}.
		\item Raumlauf mit "`Ich bin ein Träumer"'\\
			siehe \ref{subsec:Ich will} \nameref{subsec:Ich will}.
		\item Raumlauf mit "`Ich bin ein Außenseiter"'\\
			siehe \ref{subsec:Ich will} \nameref{subsec:Ich will}.
		\item Raumlauf mit "`Ich bin ein Narr"'\\
			siehe \ref{subsec:Ich will} \nameref{subsec:Ich will}.
	\end{itemize}




	\section{Textarbeit/ Rollenarbeit}
	\subsection{Archetypen}
	Alle Übungen zu den Archetypen können auch hier angewandt werden. Siehe \ref{sec:Archetypen} \nameref{sec:Archetypen}
	\subsection{Monolog schreiben und daran arbeiten}
	\begin{itemize}
		\item Zunächst schreiben die Teilnehmer einen Monolog. Die Themenwahl kann dabei komplett frei sein oder durch gewisse Rahmen eingegrenzt werden.
		\item Nun teilen die Teilnehmer den Monolog in Abschnitt ein. Ein Abschnitt endet immer, wenn sich das "`Wie"', "`Wann"', "`Wo"', "`Warum"' oder "`Wer"' des Sprechers/ Erzählers ändert.
		\item Zu den Abschnitten werden nun passende Gefühle geschrieben.
		\item Nun werden die Kernsätze/ -aussagen der Abschnitte noch aufgeschrieben.
	\end{itemize}
	\subsection{psychologische Geste finden}
	Die Gefühle, die während der Monologarbeit aufgeschrieben wurden, werden nun mit Hilfe einer \emph{psychologischen Geste} "`ausgelöst"'. Es geht dabei \textbf{nicht} darum das Gefühl \textbf{darzustellen}, sondern welche Körperhaltung, o.ä. dieses Gefühl im Teilnehmer \textbf{auslöst}.
	\subsection{Vokale/ Kernsätze mit Geste sprechen}
	In dieser Übung werden den Kernsätzen die entsprechenden \emph{Gesten} zu geteilt. Und der Teilnehmer versucht flüssig zwischen diesen zu wechseln.\\
	Als Vorübung können den \emph{Gesten} erst Vokale zu gewiesen werden. Auch bei den Vokalen geht es darum zwischen den \emph{Gesten} flüssig zu wechseln. Der entsprechende Vokal sollte dabei über den kompletten Wechselvorgang gehalten werden. Beherrscht der Teilnehmer das Wechseln, kann zu den Kernsätzen  übergegangen werden.\\
	Schließlich wird der kompletten Monolog mit den \emph{Gesten} gesprochen.
	\subsection{Raumlauf mit archetypischen Satz/ Kernsatz}
	Teilnehmer laufen zunächst mit einem archetypischen Saz (siehe \ref{subsec:Ich will} \nameref{subsec:Ich will} durch den Raum und lauschen ihrem Körper für eine psychologische Geste. Sobald diese gefunden wurde, wird das Satz durch einen Kernsatz ausgestauscht.



	\section{Welt der Sphären}
	Zunächst wird wieder das Gefühl des Zusammenziehens und Ausdehnens mit dem Ballen der Faust verbunden (Siehe \ref{sec:Zusammenziehen und Ausdehnen} \nameref{sec:Zusammenziehen und Ausdehnen}. Anschließend wird der Raum vor einem mit einem Geschmack gefüllt. Ist der Raum mit diesem Geschamck ausgefüllt, tritt der Teilnehmer in den Raum ein und lässt den Geschmack über sein Gesicht, Schulter und Haut "`fließen"' und am Ende versucht er ihn zu schmecken.
	Geschmäcker sind:
	\begin{itemize}
		\item süß
		\item sauer
		\item bitter
		\item salzig
	\end{itemize}

	\newpage

	\chapter{Stanislawski}
	\section{Körperarbeit}
	\subsection{4 Punkte im Raum}
	Bei dieser Übung sind mehrere Durchgänge erforderlich. Bei jedem neuen Durchgang kommt ein neues Körperteil dazu. \\
	Die Teilnehmer suchen sich vier Punkte im Raum (oben, unten, links, rechts). Nun werden die Punkte zunächst nur mit den Augen fixiert, dann dreht sich der Kopf, der Oberkörper, die Arme und am Ende die Beine zu den Punkten. Sobald alle Gliedmaßen den Punkt fixieren ist darauf zu achten, welche Gefühle die Punkte auslösen.

	\chapter{Meisner} %Notiz: Siehe AUfschriebe NICK!!
	Bisher ist mir die Meisner-Methode nur während meines Studiums in Form von etlichen Seminaren begegent. Der Dozent war ebenfalls professionller Schauspieler und Regiesseur. Deswegen werde ich meine bisher gesammelten Erfahrung und Aufschriebe schon einmal veröffentlichen und diese dann bei Zeiten ergänzen und ggfs. verbessern.\\
	\vspace{1cm}

	Sanford Meisner entwickelte seine Methode als Kritik zu den bestehenden Systemen, die die Darstellenden zu sehr in ihre eigene Köpfe verfrachten würden. Daher wollte er eine Methode, die kein starres enges Korsett liefert, sondern viel mehr das Spiel und seine Vielfältigkeit in den Fokus rückt. Mit dieser Voraussetzung wurden vier Grundprinzipien der Meisner-Methode formuliert.
	\begin{itemize}
		\item Discovery first
		 \item Fuck polite
		 \item Authentisches (Re)aktieren von Moment zu Moment innerhalb eines vom Text bestimmten Rahmens
		 \item Der Moment ist steuerbar aber nicht wiederholbar
	\end{itemize}
	Um vor allem das vierte Prinzip zu üben und zu ermöglichen stellt Sanford Meisner noch die fünf Bedindungen vor.
	\begin{itemize}
		\item Ziel
		\item Als ob
		\item Vorbereitung
		\item Preis und Status
		\item Externe
	\end{itemize}
	Um das dritte Prinzip
	\footnote{Das Authentische Reagieren im Moment stellte für Sanford Meisner den Unterschied zu den anderen Methoden dar. Ich behaupte, dass diese intensive Fokusierung charakteristisch für die Meisner-Methode istt}
	 zu trainieren stellt Meisner eine Übung in den Mittelpunkt seiner Methode, "`Die Wiederholung"'


	\section{Grundlagen}

	\subsection{Authentisches Reagieren}
	\subsubsection{Beobachten und Reagieren}
	Zwei Darstellende stellen sich gegenüber und beobachten den anderen. Sobald einer der beiden den Impuls hat einen Schritt vor bzw. zurück zu gehen, gibt er diesem nach. Der Zweite reagiert auf diesen Impuls.\\
	Wichtig ist dabei wirklich auf seine eigenen Impulse zu hören und keine Wertung in die Aktionen des gegenübers zu legen. 



	\setcounter{chapter}{0}
	\setcounter{section}{0}
	\part{Schauspielübungen und -spiele}
	\chapter{Schauspielübungen und -spiele}

	\section{Raumlauf}
	Alle Teilnehmer laufen durch den Raum und müssen versuchen die Lücken zu füllen. Kann mit den unterschiedlichen Bewegunstempi (0-5)\footnote{Bewegungstempi: 0 $=$ Zeitlupe; 1 $=$ etwas schneller 2 $=$gemütliches Schlendern 3 $=$ normal 4 $=$ angespannt, zielgerichtet, 5 $=$ Hektik}  gepaart werden. Kann auch mit Musik gemacht werden.
	\subsection{Raumlauf mit Kreideblase}
	Zunächst verteilen sich alle Teilnehmer im Raum und schließen die Augen. Danach zeichnen sie einen Kreideumriss um ihren Körper. Sie fangen am Kopf an, runter zu den Füßen und wieder hoch zum Kopf (Schritt und Innenschenkel werden ausgespart). Anschließend stellen sie sich vor, dass dieser Kreideumriss aufgeblasen werden kann, wie ein Ballon. Zunächst nur zur Seite, dann nach vorne und hinten und schließlich auch nach oben. Nun sollen sie mit dieser Kreideblase erneut durch den Raum laufen. Dabei müssen die Teilnehmer wieder darauf achten, dass sie die Lücken im Raum schließen. $\rightarrow$ Anschließend wird besprochen was sich verändert hat.
	\subsection{Raumlauf mit Wahrnehmung des Raumes}
	\subsection{Raumlauf mit Wahrnehmung der Personen}
	\subsection{Raumlauf mit Begrüßung}
	Alle Teilnehmer gehen durch den Raum und begrüßen sich, je nach Impuls und Laune. Dabei sollte jeder Teilnehmer jeden einmal begrüßt haben. Anschließend wird die Übung wiederholt mit der Erinnerung an die Begrüßung.
	\section{Gang zum Stuhl}
	Ein Teilnehmer versucht so einfach und so natürlich wie möglich zu einem Stuhl hinzulaufen, sich draufzusetzen und wieder abzugehen. Zwei Teilnehmer imitieren den ersten nacheinander. Anschließend wird diese Übung mit \emph{Subtext} wiederholt.
	\section{Gang zum Stuhl mit Subtext}\label{Gang zum Stuhl mit Subtext}
	Teilnehmer geht zu einem Stuhl und spricht dabei gedanklich einen Subtext. Er bleibt kurz sitzen und verlässt dann wieder die Bühne. \vspace{.4cm}\\
	\textbf{Variante} \vspace{.2cm}\\
	Ein anderer spricht den Subtext und der Spieler reagiert entsprechend.
	\section{Szene mit 3 Zügen - Etüde}
	Ähnlich \ref{Gang zum Stuhl mit Subtext}. Die Örtlichkeit und Requisiten können frei gewählt werden. Allerdings braucht der Spieler von den Auf- und Abgang einen Subtext, sowie während der Szene 3 weitere.
	\section{Tür-Impro}
	Zwei Spieler und eine Tür. Ein Spieler verlässt den Raum und überlegt sich einen Grund (Not) warum er unbedingt durch die Tür will. Der andere Spieler beschreibt kurz sein aktuelles Empfinden und das Klopfen des anderen.
	\section{Klatschkreis}
	Im Kreis werden folgenden Impulse weitergegeben:
	\begin{itemize}
		\item S\textbf{li}sch (\textbf{li}nks und klatschen)
		\item Slasch (rechts und klatschen)
		\item Boing (Arme vor Brust überkreuzen ) Wirf Impuls zurück
		\item Hah (Fruitninja) - Impuls mit gefalteten Händen werfen <Hah> $\rightarrow$ mit <Hih> annehmen $\rightarrow$ Nachbarn <Hoh> (Baum fällen) $\rightarrow$ Fänger mit <Hah> weiter geben.
	\end{itemize}
	\section{Whiskey-Mixer}
	Folgende Impulse werden im Kreis weitergegeben:
	\begin{itemize}
		\item Whiskey-Mixer (links)
		\item Wachsmaske (rechts)
		\item Messwechsel (Richtungswechsel)
	\end{itemize}
	Sobald Teilnehmer lachen müssen, müssen diese einmal um den Kreis rennen.
	\section{High-Noon}
	Alle stehen im Kreis. Es wird ein Name genannt und derjenige muss auf die Knie fallen und seine Nachbarn duellieren sich mit dem Wort <Peng>. Dabei gewinnt der, der zu erst <Peng> sagt oder es länger hält. Ist derjenige in der Mitte zu langsam wird er getroffen.\footnote{Alternative: Derjenige der getroffen wurde, scheidet aus. Sobald nur noch zwei Teilnehmer übrig sind, stellen sie sich Rücken an Rücken und laufen auseinander. Ein dritter Teilnehmer zählt Obst o.ä. auf und darunter befindet sich eine falsches Wort. Das ist das Kommando zum <Peng>-Duell für die beiden Finalisten.}
	\section{Bibedi-Bibedi-Bop}
	Teilnehmer stellen sich im Kreis auf und einer geht in die Mitte. Der Teilnehmer in der Mitte muss nun versuchen aus dem Kreis herauszukommen, indem er mit de Finger auf einen Außenstehen zeigt und das Wort "`Bibedi-Bibedi-Bop"' sagt. Der Außenstehende muss bevor der Mittlere fertig ist "`Bop"' sagen, sonst muss er in die Mitte. Zusätzliche Figuren:
	\begin{itemize}
		\item nur Bop $\rightarrow$ Außen schweigt
		\item Laterne
		\item Toaster
		\item kaputter Toaster
		\item James Bond
		\item Kotzendes Kängeru
		\item Krokodil
	\end{itemize}
	\section{Zu-Blinzeln}
	Teilnehmer stellen sich in zweier Paaren voreinander im Kreis auf. Ein Spieler bleibt allein. Nun muss er versuchen durch zublinzeln einen anderen Spieler zu "`klauen"'.
	\section{Blinzel-Mörder}
	Spielleiter wählt einen Mörder, der durch zublinzeln die anderen töten kann (10 Sekunden warten bevor man stirbt). Die anderen müssen versuchen den Mörder zu erkennen und es dem Spielleiter mitteilen. Liegen sie falsch, dann sterben auch sie.



	\setcounter{chapter}{0}
	\setcounter{section}{0}
	\part{Inszenierung}
	\chapter{Inzensierung}
	Um Stücke zu inszenieren gibt es grundsätzlich keine richtige oder falsche Methode. Wie in vielen Disziplinen führen auch hier viele Wege an Ziel. Gerade im Kinder- und Jugendtheaterbereich ist das \emph{Biographische Theater} nach \emph{Maike Plath} oder \emph{Wie hieß die andere} weit verbreitet. Bei Zeiten werden ich auch hier eine kurze Übersicht liefern. \\
	Um Kinderbücher, Märchen oder andere Geschichten, die nicht Biographien und/ oder Gedanken der Akteure aufbauen zu inszenieren, muss man sich einer anderen Technik bedienen. Mit geübten und erfahrenen Schauspielerinnen und Schauspieler ist dadurch möglich ein Stück innerhalb von wenigen Stunden zu inszenieren.

	\section{Inszenierung mittels Bilder}
	Kurze Kapitelbeschreibung

	\begin{enumerate}
		\item Geschichte lesen\\
			Zunächst solle die gesamte Gruppe die Geschichte gemeinsam lesen und mögliche Sinn- und Inhaltsfragen klären.
		\item Abschnitte einteilen\\
			Nachdem die Geschichte gelesen und verstanden wurde, beginnt man damit die Geschichte in Abschnitte einzuteilen. Ein Abschnitt oder Sequenz ändert sich immer, wenn sich auch eine Qualität\footnote{Qualitäten: Ort, Zeit, Person} ändert.
		\item Überschriften für Abschnitte finden
		\item Abschnitte in Bilder stellen
		\item Bilder überprüfen\\
			Mit Hilfe der gestellten Bilder muss nun überprüft werden, ob sich die Geschichte durch die Bilder erzählt.
		\item Szenen entwickeln
		\item Proben und Sichern
	\end{enumerate}




	\setcounter{chapter}{0}
	\setcounter{section}{0}
	\part{Atemtechnik und Sprecherziehung}
	%Notiz: Übungen nach ZWeck kategorisieren
	\chapter{Atemtechnik und Sprecherziehung}
	Das Werkzeug des Schauspielers ist neben seines Körpers, seine Stimme. Die Darstellenden müssen nicht nur darauf achten, dass sie artikulatorisch korrekt sprechen, sonder auch physiologisch angenehm. Um die Stimme zu trainieren und zu schonen, werden im folgenden verschiedene Übungen vorgestellt. Daraus kann sich jeder selbst sein persönliches Programm zusammenstellen
	\section{Übungen}
	\subsection{Pendeln}
	Leicht vor und zurück pendeln, als hätte man einen Pinsel auf dem Kopf und müsste die Decke streichen.\\
	Hier kann nun auch das Teddybärbrummen\footnote{Vokale mit <m> am Anfangen phonieren.} eingesetzt werden.
	\subsection{Trampeln}
	Mit beiden Beinen kräftig auf den Boden stampfen.\footnote{Geht auch im Sitzen.}
	\subsection{Beine/ Arme ausschütteln}
	Nach einander die Beine und die Arme ausschütteln/ schlottern lassen.\footnote{Geht auch im Sitzen}
	\subsection{Abklopfen}
	Hüftbreit hinstellen \footnote{Dazu beide Füße zusammen machen und die Versen anheben und um 45 Grad nach außen rotieren. Schließlich die Zehen nachstellen. Füße müssen parallel stehen.}, Knie leicht gebeugt und Rücken gerade. Dann beginnen die linke Schulter abzuklopfen, über den Ober- und Unterarm zur Achsel dann den Rumpf und hinterer Rücken, die Oberschenkel, die Knie mit federnden kreisenden Bewegungen lockern und und zum Schluss den Fuß in den Boden Klopfen.\\
	 Das gleiche für die andere Körperseite.\\
	 Schließlich den Kopf und Nacken ausklopfen, Wangen und Schläfen und die Kiefermuskulatur.
	 Am Ende Hände mit einem "`mmhh"' fallen lassen.
	 \subsection{Abklopfen in 3er-Gruppen}
	 Teilnehmer finden sich in 3er-Gruppen. Einer stellt sich Kopf überhängend hin und die anderen zwei Klopfen nach und nach Rücken, Arme und Beine aus. Anschließend Arme, Beine und Rücken ausstreichen. Danach mit Finger an Wirbelsäule entlang spazieren, überhängender Teilnehmer richtet sich nach und nach auf.
	 \section{Kiefer öffnen}
	 Handballen auf Wangenknochen legen. Nun schmilzt die Wärme der Hände den den Kiefer dahin (dieser besteht aus Wachs). Dabei mit einem langen und entspannten Seufzer die Hände fallen lassen. Das gleiche auch um den Kiefer nach hinten zu öffnen. Dabei halten wir allerdings auf der Höhe des Kiefergelenks an und ziehen die Hände nach hinten und entdecken dabei eine kleine Überraschung in der Kreismitte.
	 \section{Kiefer verschieben}
	 Mit Hilfe der Hände den Kiefer sanft nach links und rechts, vor und zurück schieben. Dann einen Kreis beschreiben. Den Kiefer anschließend locker lassen und mit Hilfe der Hand auf und zu machen.  Am Ende die Kieferkanten ausstreichen.
	 \section{Luftkugel}
	 Eine kleine Luftkugel in den Mund nehmen \footnote{Manchmal wird auch das Bild eine heißen Kartoffel im Mund verwendet, die man nicht berühren will.}. Dies einige Sekunden halten. Danach die Luftkugel entfernen und lächeln. Abschließend beides kombinieren zum sogenannte Sängergesicht. Fördert den vorderen Stimmsitz.
	 \section{Amulette}
	 Jeder Teilnehmer stellt sich vor, dass er ein Amulette o.ä in den Händen hält. Dieses hängt er sich um den Hals und hält es kurzzeitig noch fest. Die "`Schwere"' des Amulettes spüren und den anderen Teilnehmern präsentieren.
	 \section{Vierteilige Lockerung}
	 Auf den Zehenspitzen wippen, Knie beugen und wieder strecken, Schultern hoch- und runterziehen. Alles zusammen, kann in leichtes Springen gehen. Kiefer dabei locker lassen.
	 \section{Seufzer}
	 Beim Abklopfen Seufzer der Erleichterung fahren lassen. Ist eine Übung zur Stimmhöhenschulung.\footnote{Seufzer mit der "`Hand führen"'}
	 \subsection{Überhängender Seufzer}
	Schulterbreit hinstellen, Knie leicht gebeugt. Nun über die Schwere des Kopfes nach vorne "`abrollen"' und dann langsam Wirbel für Wirbel aufrichten. Sobald man oben ist einen Seufzer der Erleichterung "`fahren lassen"'
	\section{Kutschersitz}
	In den Kutschersitz gehen und dabei alle Gesichtsmuskeln locker lassen. Jetzt einen Testsatz sprechen.
	\section{Tonkauen}
	Ans Lieblingsessen denken, den Geruch riechen und genüsslich kauen. Und dabei ein "`mmhh"' sprechen.
	\section{Tonhöhen mit Interjektionen variieren}
	\section{Tonführung}
	Mit einer oder beiden Händen einen Nasal von hoch nach tief "`führen"'
	\section{Schulterübung}
	Schulter hochziehen bis es nicht mehr geht und mit einem "`ft"' fallen lassen.
	\section{Von Singstimme in Sprechstimme}
	Silben nim nem nam nom num "`singen"' und dann Testsatz sprechen.
	\begin{itemize}
		\item Nein, nein, nein
		\item Nein, nie, niemals
		\item Nein, das möchte ich nicht
		\item Nein, nicht mit mir
	\end{itemize}
	\section{Namen}
	 Alle Teilnehmer stellen sich im Kreis auf und werfen sich einen Ball hin und her. Dabei sagen sie immer wieder laut ihren Namen. Der Ball wird dabei mit 2 Händen auf Kopfhöhe gefangen und in einer fließenden Bewegung weiter geworfen.
	 \section{Laute sprechen}
	 Folgende Laute sollen schön betont ausgesprochen werden.
	  \begin{itemize}
	  	\item \emph{f, s, ch, sch}\\
	  	Zur Visualierung kann folgende Geschichte erzählt werden: Wir entdecken eine Feder auf der Schulter des Gegenübers {f}. Wir betrachten die Feder für uns allein {s}. Wir zeigen die Feder dem Gegenüber (Schau mal was ich habe){ch}. Wir pusten die Feder wieder zum Gegenüber zurück {sch}.
	  	\item \emph{vom s zum sch und zurück}\\
	  	Wir ziehen die Perücke vom Kopf des Gegenübers {sch} und präsentieren sie ihm. Dann zücken wir ein Feuerzeug, entzünden es {s} und verbrennen die Perücke {sch}. Anschließend lassen wir sie auf den Boden fallen {ft}.
	  \end{itemize}
	  \section{Arme heben und beim Senken Laute phonieren}
	  Arme heben und mit dem Laut <s>; <z>; <w> und <ft> nacheinander senken.
	  \section{Spannung}
	  Finger verschränken und auseinanderziehen dabei phononieren.
	  \begin{itemize}
	  \item <ft>
	  \item <jo>
	  \item <mo>
	  \item <so>
	  \end{itemize}
	  \section{Rufstimme}
	  Hände zum Trichter formen und <Hallo> rufen, als wäre man in den Bergen und würde dem Echo lauschen wollen.\footnote{Funktioniert auch im Piano.} Danach in die Sprechstimme.
	  \section{Gaumensegel entspannnen}
	  Mit Zungenspitze über den harten Gaumen bis zum weichen fahren und kurz innehalten.



	  \chapter{Persönliches Trainingsprogramm}
	  \begin{itemize}
	  	\item Trampeln
	  	\item Abklopfen
	  	\item Vierteilige Lockerung
	  	\item Kiefermuskulatur massieren und abklopfen
	  	\item Geführt Diphthonge sprechen
	  	\item Überhängender Seufzer/ Kutschersitz
	  	\item Geführtes Stimmhöhenvariation/ Sing zu Sprechstimme
	  	\item Ziehen
	  \end{itemize}





\end{document}
